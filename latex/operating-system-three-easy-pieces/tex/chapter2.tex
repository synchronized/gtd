%# -*- coding: utf-8-unix -*-
%%==================================================

\section{A Dialogue on the Book}
\noindent\textbf{Professor}: Welcome to this book! It’s called Operating
Systems in Three Easy Pieces, and I am here to teach you the things you need to
know about operating systems. I am called “Professor”; who are you? \\
\noindent\textbf{老师}:欢迎阅读本书!本书的书名是《Operating Systems in Three
Easy Pieces》,我在这里教授你们需要知道的操作系统, and I am here to teach you the things you need to
know about operating systems. I am called “Professor”; who are you? \\
\noindent\textbf{Student}: Hi Professor! I am called “Student”, as you might
have guessed. And I am here and ready to learn! \\
\noindent\textbf{Professor}: Sounds good. Any questions? \\
\noindent\textbf{Student}: Sure! Why is it called “Three Easy Pieces”? \\
\noindent\textbf{Professor}: That’s an easy one. Well, you see, there are these
great lectures on Physics by Richard Feynman... \\
\noindent\textbf{Student}: Oh! The guy who wrote “Surely You’re Joking, Mr.
Feynman”, right? Great book! Is this going to be hilarious like that book was?

\noindent\textbf{Professor}: Um... well, no. That book was great, and I’m glad
you’ve read it. Hopefully this book is more like his notes on Physics. Some of
the basics were summed up in a book called “Six Easy Pieces”. He was talking
about Physics; we’re going to do Three Easy Pieces on the fine topic of
Operating Systems. This is appropriate, as Operating Systems are about half as
hard as Physics. \\
\noindent\textbf{Student}: Well, I liked physics, so that is probably good. What
are those pieces? \\
\noindent\textbf{Professor}: They are the three key ideas we’re going to learn
about: virtualization, concurrency, and persistence. In learning about these
ideas, we’ll learn all about how an operating system works, including how it
decides what program to run next on a CPU, how it handles memory overload in a
virtual memory system, how virtual machine monitors work, how to manage
information on disks, and even a little about how to build a distributed system
that works when parts have failed. That sort of stuff. \\
\noindent\textbf{Student}: I have no idea what you’re talking about, really. \\
\noindent\textbf{Professor}: Good! That means you are in the right class. \\
\noindent\textbf{Student}: I have another question: what’s the best way to
learn this stuff? \\
\noindent\textbf{Professor}: Excellent query! Well, each person needs to figure
this out on theirown, of course, but here is what I would do: go to class, to
hear the professor introduce the material. Then, at the end of every week, read
these notes, to help the ideas sink into your head a bit better. Of course, some
time later (hint: before the exam!), read the notes again to firm up your
knowledge. Of course, your professor will no doubt assign some homeworks and
projects, so you should do those; in particular, doing projects where you write
real code to solve real problems is the best way to put the ideas within these
notes into action. As Confucius said... \\
\noindent\textbf{Student}: Oh, I know! ’I hear and I forget. I see and I
remember. I do and I understand.’ Or something like that. \\
\noindent\textbf{Professor}: (surprised) How did you know what I was going to
say?! \\
\noindent\textbf{Student}: It seemed to follow. Also, I am a big fan of
Confucius, and an even bigger fan of Xunzi, who actually is a better source for
this quote1 . \\
\noindent\textbf{Professor}: (stunned) Well, I think we are going to get along
just fine! Just fine indeed. \\
\noindent\textbf{Student}: Professor  just one more question, if I may. What are
these dialogues for? I mean, isn’t this just supposed to be a book? Why not
present the material directly? \\
\noindent\textbf{Professor}: Ah, good question, good question! Well, I think it
is sometimes useful to pull yourself outside of a narrative and think a bit;
these dialogues are those times. So you and I are going to work together to make
sense of all of these pretty complex ideas. Are you up for it? \\
S\noindent\textbf{tudent}: So we have to think? Well, I’m up for that. I mean,
what else do I have to do anyhow? It’s not like I have much of a life outside
of this book. \\
\noindent\textbf{Professor}: Me neither, sadly. So let’s get to work! 