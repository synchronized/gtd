%# -*- coding: utf-8-unix -*-
%%==================================================

\section{To Everyone}
Welcome to this book! We hope you’ll enjoy reading it as much as we enjoyed
writing it. The book is called Operating Systems: Three Easy Pieces, and the
title is obviously an homage to one of the greatest sets of lecture notes ever
created, by one Richard Feynman on the topic of Physics [F96]. While this book
will undoubtedly fall short of the high standard set by that famous physicist,
perhaps it will be good enough for you in your quest to understand what
operating systems (and more generally, systems) are all about.

欢迎阅读本书!希望读者能享受阅读它就像我们享受创作它一样。本书名为《操作系统-三
个简单部分》,并且标题显然是对有史以来最伟大的演讲--Richard Feynman在物理
主题[F96]的致敬,虽然这本书无疑将达不到那位着名物理学家所设定的高标准,但是它
也许是一本能很好的帮助你理解什么是操作系统(更一般地说,系统)的书

The three easy pieces refer to the three major thematic elements the book is
organized around: virtualization, concurrency, and persistence. In discussing
these concepts, we’ll end up discussing most of the important things an operating
system does; hopefully, you’ll also have some fun along the way. Learning new
things is fun, right? At least, it should be.

《三个简单部分》包含三个主要元素,这本书围绕:虚拟化,并发和持久化。讨论他们的概
念,最终我们会讨论操作系统大多数重要部分,希望读者一路上也会有一些乐趣。学习心的
事物是有趣的,不是吗?至少它应该是的

Each major concept is divided into a set of chapters, most of which present a
particular problem and then show how to solve it. The chapters are short, and try
(as best as possible) to reference the source material where the ideas really came
from. One of our goals in writing this book is to make the paths of history as clear
as possible, as we think that helps a student understand what is, what was, and
what will be more clearly. In this case, seeing how the sausage was made is nearly
as important as understanding what the sausage is good for1.

每一个主要概念都分为一组章节,大多数章节都会提出一个特殊的问题,然后展示如何解决
它。章节很短,试一试(尽可能做到最好)引用这些想法来源的材料。我们创作本书的目的
之一就是尽可能地让历史的路径变得清晰,我们认为这z有助于学生明白这些主题的过去,
现状以及清楚它们的未来,在这种情况下,看看香肠是如何制作的和了解香肠有什么关系一
样重要