1.Unix与Linux发展史
1)1965年,美国麻省理工学院(MIT),通用电器公司(GE),AT&T旗下贝尔实验室联合开发
Multics工程计划,其目的是开发一种交互式具有多道处理程序处理能力的分时操作系统,但
因Multics追求的目标过于庞大复杂,项目进度远远落后于计划,最后贝尔实验室宣布退出.
2)1969年,美国贝尔实验室的肯-汤姆森在DEC公司 PDP-7机器上开发出了Unix系统
    Space Travel
3)1971年,肯-汤姆森的同事丹尼斯-里奇发明C语言.
    贝尔实验室 nroff
4)1972年,贝尔实验室, 10台Unix
5)1973年,Unix系统的绝大部分源代码用C语言重写,这为提高Unix系统的可移植性打下了基础
    发明C语言(B)
6)1974年,两人在《美国计算机通信》杂志上发表论文,介绍Unix
    随后各大高校,研究机构纷纷研究Unix,在其基础上开发,并且反馈给贝尔实验室
9)1976年,肯-汤姆森在年休期间到伯克利(Berkeley)任教
    Bill Joy 成立BSRG(Berkeley System Research Group),
10)1977年,BSRG发布BSD(Berkeley System Distribution)
11)1978年,SCO公司(第一家)公开发售商业版本Unix和商业版本C编译器
12)1979年,System v7
13)1980年,Microsoft,只有两个款产品,Unix(XENIX)和basic编译器
14)1980年,美国国防部高级研究计划局(DARPA)研究tcp/ip
15)1981年,Micorsoft, Bill Gates
    QDOS(Quick and Dirty Operating System)
16)1982年,Bill Joy 成立SUN(Sun Microsystems)
    workstation(BSD)
16)1983年,tcp/ip协议诞生
2.开源软件简介
3.Linux应用
4.Linux学习方法
