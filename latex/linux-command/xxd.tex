\documentclass{ctexart}
\usepackage{changepage}
\CTEXsetup[format={\bfseries}]{section}
\begin{document}
\section*{NAME}
\begin{adjustwidth}{1cm}{0cm}
\underline{xxd} - 转换二进制数据为16进制文本,或者从16进制文本转换到二进制数据
\end{adjustwidth}

\section*{SYNOPSIS}
\begin{adjustwidth}{1cm}{0cm}
xxd -h[elp] \\
xxd [options] [infile [outfile]] \\
xxd -r[evert] [options] [infile [outfile]]
\end{adjustwidth}

\section*{DESCRIPTION}
\begin{adjustwidth}{1cm}{0cm}
\underline{xxd} creates a hex dump of a given file or standard input.  It can
also convert a hex dump back to its original binary form.  Like uuencode(1) and
uudecode(1) it allows the transmission of binary data in a `mail-safe' ASCII
representation, but has the advantage of decoding to standard output.  Moreover,
it can be used to perform binary file patching. 
\end{adjustwidth}

\section*{OPTIONS}
\begin{adjustwidth}{1cm}{0cm}
If no \underline{infile} is given, standard input is read.  If
\underline{infile} is specified as a `-' character, then input is taken from
standard input.  If no \underline{outfile} is given (or a `-' character is in
its place), results are sent to standard output. \\

\noindent Note that a ``lazy'' parser is used which does not check for more than
the first option letter, unless the option is followed by a parameter.  Spaces
between a single option letter and its parameter are optional.  Parameters to
options can be specified in decimal, hexadecimal or octal notation.  Thus -c8,
-c 8, -c 010 and -cols 8 are all equivalent.  \\

\noindent \underline{-a} | \underline{-autoskip} 
\begin{adjustwidth}{1cm}{0cm}
  toggle autoskip: A single '*' replaces nul-lines.  Default off. \\
\end{adjustwidth}

\noindent \underline{-b} | \underline{-bits} 
\begin{adjustwidth}{1cm}{0cm}
  Switch to bits (binary digits) dump, rather than hexdump.  This option writes
  octets as eight digits "1"s and "0"s instead of a normal hexadecimal dump. Each
  line is  preceded by a line number in hexadecimal and followed by an ascii (or
  ebcdic) representation. The command line switches -r, -p, -i do not work with
  this mode. \\
\end{adjustwidth}

\noindent \underline{-c} \underline{cols} | \underline{-cols} \underline{cols}
\begin{adjustwidth}{1cm}{0cm}
  format <cols> octets per line. Default 16 (-i: 12, -ps: 30, -b: 6). Max 256. \\
\end{adjustwidth}

\noindent \underline{-E} | \underline{-EBCDIC}
\begin{adjustwidth}{1cm}{0cm}
  Change the character encoding in the righthand column from ASCII to EBCDIC.
  This does not change the hexadecimal representation. The option is meaningless
  in combinations with -r, -p or -i. \\
\end{adjustwidth}

\noindent \underline{-e}
\begin{adjustwidth}{1cm}{0cm}
  Switch to little-endian hexdump.  This option treats byte groups as words in
  little-endian byte order.  The default grouping of 4 bytes may  be  changed
  using  -g.   This option only applies to hexdump, leaving the ASCII (or
  EBCDIC) representation unchanged.  The command line switches -r, -p, -i do not
  work with this mode.  \\
\end{adjustwidth}

\noindent \underline{-g} \underline{bytes} | \underline{-groupsize} \underline{bytes}
\begin{adjustwidth}{1cm}{0cm}
  separate  the  output  of  every <bytes> bytes (two hex characters or eight
  bit-digits each) by a whitespace.  Specify -g 0 to suppress grouping.  <Bytes>
  defaults to 2 in normal mode, 4 in little-endian mode and 1 in bits mode.
  Grouping does not apply to postscript or include style. \\
\end{adjustwidth}

\noindent \underline{-h} | \underline{-help}
\begin{adjustwidth}{1cm}{0cm}
  print a summary of available commands and exit.  No hex dumping is performed. \\
\end{adjustwidth}

\noindent \underline{-i} | \underline{-include}
\begin{adjustwidth}{1cm}{0cm}
  output in C include file style. A complete static array definition is written
  (named after the input file), unless xxd reads from stdin. \\
\end{adjustwidth}

\noindent \underline{-l} \underline{len} | \underline{-len} \underline{len}
\begin{adjustwidth}{1cm}{0cm}
              stop after writing <len> octets. \\
\end{adjustwidth}

\noindent \underline{-o} \underline{offset}
\begin{adjustwidth}{1cm}{0cm}
  add <offset> to the displayed file position. \\
\end{adjustwidth}

\noindent \underline{-p} | \underline{-ps} | \underline{-postscript} |
\underline{-plain}
\begin{adjustwidth}{1cm}{0cm}
  output in postscript continuous hexdump style. Also known as plain hexdump
  style. \\
\end{adjustwidth}

\noindent \underline{-r} | \underline{-revert}
\begin{adjustwidth}{1cm}{0cm}
  reverse operation: convert (or patch) hexdump into binary.  If not writing to
  stdout, xxd writes into its output file without truncating it. Use the
  combination -r  -p  to read plain hexadecimal dumps without line number
  information and without a particular column layout. Additional Whitespace and
  line-breaks are allowed anywhere. \\
\end{adjustwidth}

\noindent \underline{-seek} \underline{offset}
\begin{adjustwidth}{1cm}{0cm}
  When used after -r: revert with <offset> added to file positions found in
  hexdump. \\
\end{adjustwidth}

\noindent \underline{-s} \underline{[+][-]seek}
\begin{adjustwidth}{1cm}{0cm}
  start at <seek> bytes abs. (or rel.) infile offset.  + indicates that the seek
  is relative to the current stdin file position (meaningless when not reading
  from stdin).  - indicates that the seek should be that many characters from
  the end of the input (or if combined with +: before the current stdin file
  position).  Without -s  option,  xxd starts at the current file position. \\
\end{adjustwidth}

\noindent \underline{-u}
\begin{adjustwidth}{1cm}{0cm}
  use upper case hex letters. Default is lower case. \\
\end{adjustwidth}

\noindent \underline{-v} | \underline{-version}
\begin{adjustwidth}{1cm}{0cm}
  show version string. \\
\end{adjustwidth}
\end{adjustwidth}

\section*{CAVEATS}
\begin{adjustwidth}{1cm}{0cm}
\underline{xxd}  \underline{-r}  has  some  builtin  magic while evaluating line
number information.  If the output file is seekable, then the linenumbers at the
start of each hexdump line may be out of order, lines may be missing, or
overlapping. In these cases xxd will lseek(2) to the next position. If the
output file is not seekable, only gaps  are  allowed,  which  will  be filled by
null-bytes. \\ 

\noindent \underline{xxd} \underline{-r} never generates parse errors. Garbage
is silently skipped. \\  

\noindent When  editing hexdumps, please note that \underline{xxd}
\underline{-r} skips everything on the input line after reading enough columns
of hexadecimal data (see option -c). This also means, that changes to the
printable ascii (or ebcdic) columns are always ignored. Reverting a plain (or
postscript) style hexdump with xxd -r -p does not depend on the correct  number
of  columns. Here anything that looks like a pair of hex-digits is interpreted.
\\ 

\noindent Note the difference between \\
\% xxd -i file \\
and \\
\% xxd -i < file \\

\noindent \underline{xxd} \underline{-s} \underline{+seek} may be different from
\underline{xxd} \underline{-s} \underline{seek}, as lseek(2) is used to
``rewind'' input.  A `+' makes a difference if the input source is stdin, and if
stdin's file position is not at the start of the file by the time xxd is started
and given its input.  The following examples may help to clarify (or further
confuse!)$\cdots$ \\ 

\noindent Rewind stdin before reading; needed because the `cat' has already read
to the end of stdin. \\   
\% sh -c ``cat > plain\_copy; xxd -s 0 > hex\_copy'' < file \\

\noindent Hexdump from file position 0x480 (=1024+128) onwards.  The `+' sign
means "relative to the current position", thus the `128' adds to the 1k where dd
left off. \\ 
\% sh -c ``dd of=plain\_snippet bs=1k count=1; xxd -s +128 > hex\_snippet'' <
file \\ 

\noindent Hexdump from file position 0x100 ( = 1024-768) on. \\
\underline{\% sh -c ``dd of=plain\_snippet bs=1k count=1; xxd -s +-768 >
  hex\_snippet'' < file} \\ 

\noindent However, this is a rare situation and the use of `+' is rarely needed.
The author prefers to monitor the effect of xxd with strace(1) or truss(1),
whenever -s is used. 
\end{adjustwidth}

\end{document}