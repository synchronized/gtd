\documentclass[a4paper,11pt]{ctexart}
\usepackage{amsmath}
\usepackage{amssymb}

\CTEXsetup[format={\Large\bfseries}]{section}
\begin{document}
\section{指数和下标}
指数和下标可以用\^和\_后加相应字符来实现。比如:\\
$a_{1}$ \qquad
$x^{2}$ \qquad
$e^{-alpha t}$ \qquad
$a^{3}_{ij}$  \qquad
$e^{x^2} \neq {e^x}^2$ \qquad
$e^{x^2} \neq (e^x)^2$

\section{根号}
平方根(square root)的输入命令为:$\backslash$sqrt,
n 次方根相应地为: $\backslash$sqrt[n]。
方根符号的大小由LATEX自动加以调整。也可用$\backslash$surd 仅给出符号: \\
$\sqrt{x}$ \qquad
$\sqrt{x^{2}+\sqrt{y} }$ \qquad
$\sqrt[3]{2}$   \qquad
\\[3pt]$\surd[x^2 + y^2]$ 

\section{上划线,下划线}
命令$\backslash$overline 和$\backslash$underline 在表达式的上、下方画出水平线。比如: \\
$\overline{m+n}$ \qquad
$\underline{m-n}$

\section{垂直方向大括号}
命令$\backslash$overbrace 和$\backslash$underbrace 在表达式的上、下方给出一水平的大括号。\\
$\overbrace{ a+b+\cdots+z }^{26}$ \qquad
$\underbrace{ a+b+\cdots+z }_{26}$

\section{向量}
向量(Vectors)通常用上方有小箭头(arrow symbols)的变量表示。
这可由 $\backslash$vec 得到。
另两个命令$\backslash$overrightarrow 和$\backslash$overleftarrow在定义从A 到B 的向量时非常有用。\\
\begin{displaymath}
\vec a \qquad
\overrightarrow{AB} \qquad
\overleftarrow{MN}
\end{displaymath}

\section{分式}
分数(fraction)使用$\backslash$frac\{...\}\{...\} 排版。
一般来说,1/2 这种形式更受欢迎,因为对于少量的分式,它看起来更好些。\\
$1\frac{1}{2}Hour$ \qquad
\begin{displaymath}
  \frac{x^2}{k+1} \qquad
  \frac{2}{x^{k+1}} \qquad
  x^{\frac{1}{2}}
\end{displaymath}

\section{积分运算符}
积分运算符(integral operator)用$\backslash$int 来生成。
求和运算符(sum operator)由$\backslash$sum 生成。
乘积运算符(product operator)由$\backslash$prod 生成。上限和下限用\^ 和\_来生成,类似于上标和下标。\\
\begin{displaymath}
\sum_{i=1}^{n} \qquad
\int_{0}^{\frac{\pi}{2}} \qquad
\prod_{\epsilon}
\end{displaymath}

\end{document}
