\documentclass{ctexart}

\CTEXsetup[format={\bfseries}]{section}
\usepackage{tcolorbox}

\begin{document}
\section{Einstein}
\ldots when Einstein introduced his formula
\begin{equation}
  e = m \cdot c^2 \; ,
\end{equation}
which is at the same time the most widely known
and the least well understood physical formula.
% Example 2
\ldots from which follows Kirchoff’s current law:
\begin{equation}
  \sum_{k=1}^{n} I_k = 0 \; .
\end{equation}
Kirchhoff’s voltage law can be derived \ldots
% Example 3
\ldots which has several advantages.
\begin{equation}
  I_D = I_F - I_R
\end{equation}
is the core of a very different transistor model. \ldots

\section{特殊字符}
\subsection{保留符号}
下面的这些字符是 LATEX 的保留字符,它们或在 LATEX 中有特定的用
处,或不一定包含在所有的字库中。如果你直接在文本中使用它们,通常
在排版结果中将不会得到这些字符,而且还会导致 LATEX 做一些你不希望
发生的事情。 \\*
\# \qquad
\$ \qquad
\% \qquad
\^ \qquad
\& \qquad
\_ \qquad
\{ \qquad
\} \qquad
\~ \qquad
$\backslash$ 

当然,这些字符前面加上反斜线,就可以在文本中得到它们。 \\*

$\backslash$\# \qquad
$\backslash$\$ \qquad
$\backslash$\% \qquad
$\backslash$\^ \qquad
$\backslash$\& \qquad
$\backslash$\_ \qquad
$\backslash$\{ \qquad
$\backslash$\} \qquad
$\backslash$\~

\subsection{简单\LaTeX{}命令}
\today \\
\TeX \\
\LaTeX \\
\LaTeXe 

\subsection{引号}
这是``双引号'' \qquad
这是`单引号' \qquad
`这是什么?'

\subsection{破折号和连字号}
daughter-in-law, X-rated\\
pages 13--67\\
yes---or no? \\
$0$, $1$ and $-1$

\subsection{波浪号}
波浪号经常和网址用在一起。它在 LATEX 中,可用 \~ 产生,但其结
果:却不是你真正想要的。试一下这个:\\
http://www.baidu.com/\~{}bush \\
http://www.baidu.com/$\sim$demo 

\subsection{度的符号}
在LATEX中如何排度的符号? \\
Its $-30\,^{\circ}\mathrm{C}$,
I will soon start to
super-conduct.

\subsection{省略号}
在打字机上,逗号或句号占据的空间和其他字母相等。在书籍印刷
中,这些字符仅占据一点儿空间,并且与前一个字母贴得非常紧。所以不
能只键入三个点来输出‘省略号’,因为间隔划分得不对。有一个专门的命令
输出省略号。它称为($\backslash$ldots)

\ldots

\subsection{注音符号和特殊字符}
\LaTeX 支持来自许多语言中的注音符号和特殊字符。表 2.1 就字母 o 列
出了所有的注音符号。对于其他字母也自然有效。\\
在字母 i 和 j 上标一个注音符号,它的点儿必须去掉。这个可由 $\backslash$i 和
$\backslash$j 做到。
\\
H\^otel, na\"\i ve, \’el\‘eve,\\
sm\o rrebr\o d, !‘Se\~norita!,\\
Sch\"onbrunner Schlo\ss{}
Stra\ss e

\section{表格}
tabular 环境能用来排印带有水平和铅直表线的漂亮表格。LATEX 自动
确定每一列的宽度。\\
命令\\
\begin{tabular}{|l|}
  \hline
  % \verb-\begin{tabular}{table spec}- \\
  \hline
\end{tabular}

的参量 table spec 定义了表格的式样。用一个 l 产生左对齐的列,用一个 r
产生右对齐的列,用一个 c 产生居中的列;用 p\{宽度值width\} 产生相应宽
度、包含自动断行文本的列;| 产生铅直表线。

\begin{tabular}{|l|r|}
\hline
7C0 & hexadecimal \\
3700 & octal \\ \cline{2-2}
11111000000 & binary \\
\hline \hline
1984 & decimal \\
\hline
\end{tabular}


\begin{tabular}{|p{4.7cm}|}
\hline
Welcome to Boxy’s paragraph.
We sincerely hope you’ll
all enjoy the show. \\
\hline
\end{tabular}

\begin{tabular}{|c r @{.} l|}
  \hline
  Pi expression      & \multicolumn{2}{c}{Values} \\
  $\pi$              & 3&1416 \\
  $\pi^{\pi}$        & 36&46 \\
  $(\pi^{\pi})^{\pi}$ & 80662&7 \\
  \hline
\end{tabular}

\section{环境}
\subsection{Itemize, Enumerate, and Description}
itemize 环境用于简单的列表,enumerate 环境用于带序号的列表,description
环境用于带描述的列表。
\flushleft
\begin{enumerate}
\item You can mix the list
environments to your taste:
\begin{itemize}
\item But it might start to
look silly.
\item[-] With a dash.
\end{itemize}
\item Therefore remember:
\begin{description}
\item[Stupid] things will not
become smart because they are
in a list.
\item[Smart] things, though, can be
presented beautifully in a list.
\end{description}
\end{enumerate}

\subsection{Flushleft, Flushright, and Center}
flushleft 和 flushright 环境分别产生靠左排列和靠右排列的段
落。center 环境产生居中的文本。如果你不输入命令 $\backslash\backslash$ 指定断行点,\LaTeX
将自行决定。
\fbox{%
  \parbox{\textwidth}{%
    \begin{flushleft}
      This text is\\ left-aligned.
      \LaTeX{} is not trying to make
      each line the same length.
    \end{flushleft}
  }%
}
\fbox{%
  \parbox{\textwidth}{%
    \begin{flushright}
      This text is right-\\aligned.
      \LaTeX{} is not trying to make
      each line the same length.
    \end{flushright}
  }%
}
\fbox{%
  \parbox{\textwidth}{%
    \begin{center}
      At the centre\\of the earth
    \end{center}
  }%
}

\begin{tcolorbox}
This is another \textbf{tcolorbox}.
\tcblower
Here, you see the lower part of the box.
\end{tcolorbox}

\subsection{Quote, Quotation, and Verse}
quote 环境对重要断语和例子的引用很重要。
\fbox{%
  \parbox{\textwidth}{%
    A typographical rule of thumb
    for the line length is:
    \begin{quote}
      On average, no line should
      be longer than 66 characters.
    \end{quote}
    This is why \LaTeX{} pages have
    such large borders by default and
    also why multicolumn print is
    used in newspapers.
  }%
}

有两个类似的环境:quotation 和 verse 环境。quotation 环境用于
超过几段的较长引用,因为它对段落进行缩进。verse 环境用于诗歌,在诗
歌中断行很重要。在一行的末尾用 \\ 断行,在每一段后留一空行。
\fbox{%
  \parbox{\textwidth}{%
    I know only one English poem by
    heart. It is about Humpty Dumpty.
    \begin{flushleft}
      \begin{quotation}
        Humpty Dumpty sat on a wall:\\
        Humpty Dumpty had a great fall.\\
        All the King’s horses and all
        the King’s men\\
        Couldn’t put Humpty together
        again.
      \end{quotation}
    \end{flushleft}
  }%
}

\fbox{%
  \parbox{\textwidth}{%
    I know only one English poem by
    heart. It is about Humpty Dumpty.
    \begin{flushleft}
      \begin{verse}
        Humpty Dumpty sat on a wall:\\
        Humpty Dumpty had a great fall.\\
        All the King’s horses and all
        the King’s men\\
        Couldn’t put Humpty together
        again.
      \end{verse}
    \end{flushleft}
  }%
}

\subsection{fbox}
\fbox{%
  \parbox{\textwidth}{%
    % \begin{center}
      aaa\\
      bbb
    % \end{center}
  }%
}

\section{边框}
\subsection{tcolorbox}


\end{document}
