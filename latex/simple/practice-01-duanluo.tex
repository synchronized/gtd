\documentclass{ctexart}

\CTEXsetup[format={\bfseries}]{section}

\begin{document}
\section{Einstein}
\ldots when Einstein introduced his formula
\begin{equation}
  e = m \cdot c^2 \; ,
\end{equation}
which is at the same time the most widely known
and the least well understood physical formula.
% Example 2
\ldots from which follows Kirchoff’s current law:
\begin{equation}
  \sum_{k=1}^{n} I_k = 0 \; .
\end{equation}
Kirchhoff’s voltage law can be derived \ldots
% Example 3
\ldots which has several advantages.
\begin{equation}
  I_D = I_F - I_R
\end{equation}
is the core of a very different transistor model. \ldots

\section{特殊字符}
\subsection{保留符号}
下面的这些字符是 LATEX 的保留字符,它们或在 LATEX 中有特定的用
处,或不一定包含在所有的字库中。如果你直接在文本中使用它们,通常
在排版结果中将不会得到这些字符,而且还会导致 LATEX 做一些你不希望
发生的事情。 \\*
\# \qquad
\$ \qquad
\% \qquad
\^ \qquad
\& \qquad
\_ \qquad
\{ \qquad
\} \qquad
\~ \qquad
$\backslash$ 

\\
当然,这些字符前面加上反斜线,就可以在文本中得到它们。 \\*
$\backslash$\# \qquad
$\backslash$\$ \qquad
$\backslash$\% \qquad
$\backslash$\^ \qquad
$\backslash$\& \qquad
$\backslash$\_ \qquad
$\backslash$\{ \qquad
$\backslash$\} \qquad
$\backslash$\~

\subsection{简单\LaTeX{}命令}
\today \\
\TeX \\
\LaTeX \\
\LaTeXe 

\subsection{引号}
这是``双引号'' \qquad
这是`单引号' \qquad
`这是什么?'

\subsection{破折号和连字号}
daughter-in-law, X-rated\\
pages 13--67\\
yes---or no? \\
$0$, $1$ and $-1$

\subsection{波浪号}
波浪号经常和网址用在一起。它在 LATEX 中,可用 \~ 产生,但其结
果:却不是你真正想要的。试一下这个:\\
http://www.baidu.com/\~{}bush \\
http://www.baidu.com/$\sim$demo 

\subsection{度的符号}
在LATEX中如何排度的符号? \\
Its $-30\,^{\circ}\mathrm{C}$,
I will soon start to
super-conduct.

\subsection{省略号}
在打字机上,逗号或句号占据的空间和其他字母相等。在书籍印刷
中,这些字符仅占据一点儿空间,并且与前一个字母贴得非常紧。所以不
能只键入三个点来输出‘省略号’,因为间隔划分得不对。有一个专门的命令
输出省略号。它称为($\backslash$ldots)

\ldots

\section{表格}
tabular 环境能用来排印带有水平和铅直表线的漂亮表格。LATEX 自动
确定每一列的宽度。\\
命令\\
\begin{tabular}{|l|}
  \hline
  \verb|\begin{tabular}{table spec}| \\
  \hline
\end{tabular}

的参量 table spec 定义了表格的式样。用一个 l 产生左对齐的列,用一个 r
产生右对齐的列,用一个 c 产生居中的列;用 p\{宽度值width\} 产生相应宽
度、包含自动断行文本的列;| 产生铅直表线。

\begin{tabular}{|l|r|}
\hline
7C0 & hexadecimal \\
3700 & octal \\ \cline{2-2}
11111000000 & binary \\
\hline \hline
1984 & decimal \\
\hline
\end{tabular}

\begin{tabular}{|p{4.7cm}|}
\hline
Welcome to Boxy’s paragraph.
We sincerely hope you’ll
all enjoy the show. \\
\hline
\end{tabular}

  \begin{tabular}{|c r @{.} l|}
    \hline
    Pi expression      & \multicolumn{2}{c}{Values} \\
    $\pi$              & 3&1416 \\
    $\pi^{\pi}$        & 36&46 \\
    $(\pi^{\pi})^{\pi}$ & 80662&7 \\
    \hline
  \end{tabular}

\end{document}
